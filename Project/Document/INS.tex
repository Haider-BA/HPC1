%========================================================================
\documentclass[11pt]{article}
\usepackage{graphics}
\graphicspath{{figs/}}

%===========================================
% Fix the margins
\setlength{\topmargin}{-0.4in}    % distance from top of page to begining of text
\setlength{\textheight}{9.0in}    % height of main text
\setlength{\textwidth}{6.5in}     % width of text
\setlength{\oddsidemargin}{0.in}  % odd page left margin
\setlength{\evensidemargin}{0.in} % even page left margin

%===========================================
\usepackage{amssymb,amsfonts}
\usepackage{amsmath}

%===========================================
\begin{document}

\title{incompressible Navier Stokes equation using fractional step method}

\author{Zhengjiang Li}

\date{}

\maketitle

\section {Incompressible Navier-Stokes equation}

	$$ \frac{\partial u}{\partial t} + ( u \cdot \nabla) u = - \nabla p + \frac{1}{Re} \nabla^2 u $$

	$$ \nabla \cdot u = 0 $$


discretization with a staggered mesh finite volume formulation or with lumped finite element formulation, using implicit Crank-Nicolson(CN) integration for the viscous terms and the explicit second-order Adams-Bashforth(AB2) scheme for the convective terms, namely:

\begin{align*}
  \frac{v^{n+1} - v^n}{\Delta t} + [\frac{3}{2} H(v^n) - \frac{1}{2} H(v^{n-1}] \\
 = -G p^{n+1} + \frac{1}{2Re} L(v^{n+1} + v^n) + bc' 
\end{align*}

$$ D v^{n+1} = 0 + bc'' $$

in algebraic system :
$$ 
\begin{bmatrix} A && G \\ D && 0 \end{bmatrix} \begin{pmatrix} v^{n+1} \\ p^{n+1} \end{pmatrix} = \left( \begin{array}{c} r \\ 0 \end{array} \right) + \left ( \begin{array}{c} bc' \\ bc'' \end{array} \right) $$

where

$$ A = \frac{1}{\Delta t} [ I - \frac{\Delta t}{2Re}L] $$

$$ r = \frac{1}{\Delta t} [ I + \frac{\Delta t}{2Re} L] v^n - [\frac{3}{2} H(v^n) - \frac{1}{2} H(v^{n-1}] $$

For a algebraic system of equation above, usually we can approximate the divergence equation(the second block equation), or we can approximate the momentum equation by approximation the pressure term, as 

 $$ 
\begin{bmatrix} A &&  (AB)G \\ D &&  0 \end{bmatrix} \begin{pmatrix} v^{n+1} \\ p^{n+1} \end{pmatrix} = \left( \begin{array}{c} r \\ 0 \end{array} \right) + \left ( \begin{array}{c} bc' \\ bc'' \end{array} \right) $$

	
Take block LU decomposition will obtain:

$$ 
\begin{bmatrix} A && 0 \\ D && -DBG \end{bmatrix} \begin{pmatrix} v^* \\ p^{n+1} \end{pmatrix} = \left( \begin{array}{c} r \\ 0 \end{array} \right) + \left ( \begin{array}{c} bc' \\ bc'' \end{array} \right) $$

$$ 
\begin{bmatrix} I &&  BG \\ 0 && I \end{bmatrix} \begin{pmatrix} v^{n+1} \\ p^{n+1} \end{pmatrix} = \left( \begin{array}{c} v^* \\ p^{n+1} \end{array} \right) $$

or in the following sequence :

\begin{equation}
	A v^* = r + bc';
	DBG p^{n+1} = D v^* - bc'' ;
  	v^{n+1} = v* - BG p^{n+1};
\end{equation}

if B is chosen equal to $\Delta t$ times the identity matrix, then it give the first-order block LU decomposition;
if B is cosen to be an approximate inverse of A, then higher order accuray can be achieved.


\section {Reference}
1 An analysis of the fractional step method, J. B. Perot 1993
2 Analysis of an exact fractional step method, W. Chang 2002

\end{document}
